%!TEX root = NeedleReference.tex
\chapter{Configuration}

\section{Requirements}

Ensure that you have a 
\href{http://www.oracle.com/technetwork/java/javase/downloads/index.html}{JDK6+} installed.

\section{Maven dependency configuration}
If you are using \href{http://maven.apache.org/}{Maven} as your build tool add the following single dependency to your pom.xml file to get started with Needle:

\begin{lstlisting}[language=,caption=The Maven pom.xml,escapechar=|,label=lst:pom.xml]
   <dependency>
      <groupId>de.akquinet.jbosscc</groupId>
      <artifactId>jbosscc-needle</artifactId>
      <scope>test</scope>
      <version>${needle.version}</version>
   </dependency>
\end{lstlisting}

Where \verb|needle.version| currently should have the value "2.0.1". Check for the most current version at \href{http://needle.sourceforge.net}{our web site}.

To reduce complexity Needle has no transitive dependencies, and does thus not restrict you to use specific versions of JUnit or TestNG.
On the other hand, you will have to explicitely configure dependencies for the test scope, for example, to Hibernate as the JPA provider.

\section{Needle configuration properties}

The Needle default configuration may be modified in a \verb|needle.properties| file in the classpath root. I.e., Needle will look for a file 
\verb|/needle.properties| somewhere in the classpath.

\parskip 14pt
\parindent 0pt
Configuration of additional custom injection annotations and injection provider.

\begin{table}[H]
\centering
\begin{tabular}{p{6cm}p{11cm}} \toprule
    \textbf{Property name} & \textbf{Description} \\ \midrule
    custom.injection.annotations & Comma separated list of the fully qualified name of the annotation classes. A standard mock provider will be created for each annotation. \\
    custom.injection.provider.classes & Comma separated list of the fully qualified name of the injection provider implementations. \\
\bottomrule
\end{tabular}
\caption{Custom Injection Provider}
\end{table}

\parskip 14pt
\parindent 0pt
Configuration of mock provider.

\begin{table}[H]
\centering
\begin{tabular}{p{6cm}p{11cm}} \toprule
    \textbf{Property name} & \textbf{Description} \\ \midrule
    mock.provider & The fully qualified name of an implementation of the MockProvider interface. There is an implementation of EasyMock de.akquinet.jbosscc.needle.mock.EasyMockProvider and Mockito de.akquinet.jbosscc.needle.mock.MockitoProvider. \textbf{EasyMock is the default configuration.}  \\
\bottomrule
\end{tabular}
\caption{Mock Provider}
\end{table}


\parskip 14pt
\parindent 0pt
Configuration of JPA, Database operation and JDBC connection.

\begin{table}[H]
\centering
\begin{tabular}{p{6cm}p{11cm}} \toprule
    \textbf{Property name} & \textbf{Description} \\ \midrule
    persistenceUnit.name & The persistence unit name. Default is TestDataModel  \\
    hibernate.cfg.filename & XML configuration file to configure Hibernate (eg. /hibernate.cfg.xml)  \\
    db.operation & Optional database operation on test setup and tear down. Value is the fully qualified name of an implementation of the AbstractDBOperation base class.
There is an implementation for script execution de.akquinet.jbosscc.needle.db.operation.\-ExecuteScriptOperation and for the  HSQL DB to delete all tables de.akquinet.jbosscc.needle.db.operation.hsql.\-HSQLDeleteOperation.    \\
    jdbc.url & The JDBC driver specific connection url.  \\
    jdbc.driver & The fully qualified class name of the driver class.  \\
    jdbc.user &  The JDBC user name used for the database connection.  \\
    jdbc.password & The JDBC password used for the database connection.  \\
\bottomrule
\end{tabular}
\caption{JPA and JDBC Configuration}
\end{table}

The JDBC configuration properties are only required if database operation and JPA 1.0 are used.
Otherwise, the JDBC properties are related to the standard property names of JPA 2.0.

\subsection{Example configuration}

A typical \verb|needle.properties| file might look like this:

\begin{lstlisting}[language=,caption=needle.properties,escapechar=|,label=lst:needle.properties]
db.operation=de.akquinet.jbosscc.needle.db.operation.hsql.HSQLDeleteOperation
jdbc.driver=org.hsqldb.jdbcDriver
jdbc.url=jdbc:hsqldb:mem:memoryDB
jdbc.user=sa
jdbc.password=
\end{lstlisting}

\section{Logging}

Needle uses the Simple Logging Facade for Java (SLF4J). \href{http://www.slf4j.org/manual.html}{SLF4J} serves as a simple facade or 
abstraction for various logging frameworks.
The SLF4J distribution ships with several JAR files referred to as "SLF4J bindings", with each binding corresponding to a supported framework.

For logging within the test, the following optional dependency may be added to the classpath:

\begin{lstlisting}[language=,caption=SLF4J dependency,escapechar=|,label=lst:SLF4J]
   <dependency>
      <groupId>org.slf4j</groupId>
      <artifactId>slf4j-simple</artifactId>
      <scope>test</scope>
   </dependency>
\end{lstlisting}

