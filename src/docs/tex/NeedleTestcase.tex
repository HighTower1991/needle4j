%!TEX root = NeedleReference.tex
\chapter{Needle Testcase}

\section{ObjectUnderTest instantiation and initialization}
\label{sec: ObjectUnderTest}

Needle may automatically instantiate all objects under test for you.
The Needle test case scans all fields of the test class for annotations and creates a completely initialized instance. Alternatively, the 
developer may of course instantiate the field herself, too.

Multiple fields can be annotated with the \verb|@ObjectUnderTest| annotation. The annotation can optionally be configured with the implementation of the type and an id.
The id may be used for additional injections (explained later). When an object under test is already instantiated, only the dependency injection will be done.

\subsection{PostConstruct lifecycle callback}
\label{sec:PostConstruct}

During the object under test instantiation it is possible to execute lifecycle methods annotated with PostConstruct. The PostConstruct annotation is used on a method that needs to be executed after dependency injection is done to perform any initialization. By default lifecycle methods are ignored. The execution can be activated by the annotation @ObjectUnderTest.

\begin{lstlisting}[language={JAVA},caption=PostConstruct]
@Stateless
public class UserDao {

   @PostConstruct
   public void init(){
      ...
   }
}
\end{lstlisting}

\begin{lstlisting}[language={JAVA},caption=PostConstruct Testcase]
public class UserDaoTest {
   @Rule
   public NeedleRule needleRule = new NeedleRule();

   @ObjectUnderTest(postConstruct = true)
   private UserDao userDao;

   @Test
   public void test(){
      ...
   }	
}
\end{lstlisting}

Note: The behavior in an inheritance hierarchy is not defined by the common annotations specification.

\section{Injection}
\label{sec:Injection}

Needle supports field, constructor and method injection by evaluating \verb|@EJB|, \verb|@Resource|, \verb|@PersistenceContext|, \verb|@PersistenceUnit|
and \verb|@Inject| annotations. Note that the annotation class needs to be available one the classpath of the test execution.
By default, Mock objects for the dependencies are created and injected.

The injected (mock or real) objects can be also injected into your testcase. Test case injection can be done in the same way as the injection of dependencies in the production code. I.e., you may write \verb|@Inject| within your test case to access
the generated components.

All standard Java EE annotations and any additionally configured injection annotation can be used to inject a reference into the testcase.

\begin{lstlisting}[language={JAVA},caption=Testcase injection]
public class AuthenticatorTest {
   @Rule
   public NeedleRule needleRule = new NeedleRule();

   @ObjectUnderTest
   private Authenticator authenticator = new Authenticator();

   @Inject
   private EasyMockProvider mockProvider;

   @Inject
   private UserDao userDaoMock;

   @Test
   public void testAuthenticate() {
      ...
   }
}
\end{lstlisting}

It is also possible to use the API to get a reference of an injected object.

\begin{lstlisting}[language={JAVA},caption=Injected Components]
UserDao injectedUserDao = needleRule.getInjectedObject(UserDao.class);
\end{lstlisting}

The key is generated from the respective injection provider. By default, the class object of the associated injection point is used as the key or
-- in the case of resource injection -- the mapped name of the resource.

\section{Custom injection provider}
\label{sec:Custom injection provider}

Needle is fully extensible, you may implement your own injection providers or register additional annotations.

The following example shows the registration of additional annotations in the \verb|needle.properties|.

\begin{lstlisting}[language={JAVA},caption=Additional Annotation]
custom.injection.annotations=org.jboss.seam.annotations.In, org.jboss.seam.annotations.Logger
\end{lstlisting}

It is also possible to implement custom providers.
A custom injection provider must implement the \verb|de.akquinet.jbosscc.needle.injection.InjectionProvider| interface.

\begin{lstlisting}[language={JAVA},caption=javax.inject.Qualifier Injection Provider]
public class CurrentUserInjectionProvider implements InjectionProvider<User> {
   private final User currentUser = new User();

   @Override
   public User getInjectedObject(Class<?> injectionPointType) {
      return currentUser;
   }

   @Override
   public boolean verify(InjectionTargetInformation information) {
      return information.isAnnotationPresent(CurrentUser.class);
   }

   @Override
   public Object getKey(InjectionTargetInformation information) {
      return CurrentUser.class;
   }
}
\end{lstlisting}

A custom injection provider can be provided for a specific test or as a global provider (again in the \verb|needle.properties| file).

\begin{lstlisting}[language={JAVA},caption=Custom injection provider for a specific test]
@Rule
public NeedleRule needleRule = new NeedleRule(new CurrentUserInjectionProvider());
\end{lstlisting}

\begin{lstlisting}[language={JAVA},caption=Global custom injection provider]
custom.injection.provider.classes=de.akquinet.CurrentUserInjectionProvider
\end{lstlisting}

If you need to configure multiple InjectionProviders at once, it is possible to use the InjectionProviderInstancesSupplier. It returns
a Set of InjectionProviders.
A custom injection provider supplier must implement the \verb|de.akquinet.jbosscc.needle.injection.InjectionProviderInstancesSupplier| interface.

\begin{lstlisting}[language={JAVA},caption=Injection Provider InstancesSupplier]
public class FooBarInjectionProviderInstancesSupplier implements InjectionProviderInstancesSupplier {
   private final Foo foo = new Foo();
   private final Bar bar = new Bar();
   
   private final Set<InjectProvider<?>> providers = new HashSet<InjectionProvider<?>>();
   
   public FooBarInjectionProviderInstancesSupplier() {
      providers.add(InjectionProviders.providerForInstance(foo));
      providers.add(InjectionProviders.providerForInstance(bar));
   }

   @Override
   public Set<InjectProvider<?>> get() {
      return providers;
   }
}
\end{lstlisting}

A custom injection provider instances supplier can be provided for a specific test or also as a global provider (again in the \verb|needle.properties| file).

\begin{lstlisting}[language={JAVA},caption=Custom injection supplier for a specific test]
@Rule
public NeedleRule needleRule = NeedleRuleBuilder.needleRule().addSupplier(new FooBarInjectionProviderInstancesSupplier());
\end{lstlisting}

\begin{lstlisting}[language={JAVA},caption=Global custom injection provider supplier]
custom.intances.supplier.classes=de.akquinet.FooBarInjectionProviderInstancesSupplier
\end{lstlisting}

\section{Wiring of object dependencies}
\label{sec:Wiring of object dependencies}

Sometimes you may want to provide your own objects as components or influence the wiring of complex object graphs. Typically, you may want to mix
mock objects and "real" objects, depending on the focus of your test.

The object referenced by the field annotated with \verb|@InjectIntoMany| is injected into all fields annotated with \verb|@ObjectUnderTest|.

In the following example UserDao is not a mock, but a fully initialized component. It is injected into the Authenticator object (or any other
component annotated with \verb|@ObjectUnderTest|). 

\begin{lstlisting}[language={JAVA},caption=InjectIntoMany]
@ObjectUnderTest
private Authenticator authenticator;

@InjectIntoMany
@ObjectUnderTest
private UserDao userDao;
\end{lstlisting}

If you need more fine-grained configuration you may also use the \verb|@InjectInto| annotation to specify the component IDs where the object 
shall be injected.
The ID corresponds to the field name of the target object by default.

\begin{lstlisting}[language={JAVA},caption=InjectInto]
@ObjectUnderTest
private Authenticator authenticator;

@InjectInto(targetComponentId="authenticator")
@ObjectUnderTest
private UserDao userDao;
\end{lstlisting}





