%!TEX root = NeedleReference.tex
\chapter{Testing with Mock objects}
Mock objects are a useful way to write unit tests for objects that have collaborators. 
Needle generates Mock objects dynamically for dependencies of the components under test by default.
Out-of-the-box Needle has implementations for EasyMock and Mockito.
To use other mock frameworks, the interface \verb|de.akquinet.jbosscc.needle.mock.MockProvider| must be implemented and configured in the \verb|needle.properties| file.

\section{Create a Mock Object}

To create a Mock object, you can annotate a field with the annotation \verb|@Mock|.

\begin{lstlisting}[language={JAVA},caption=Mock annotation]
public class Test {
   @Rule
   public NeedleRule needleRule = new NeedleRule();

   @Mock
   private EntityManager entityManagerMock;

   @Test
   public void test() throws Exception {
       ...
   }
}
\end{lstlisting}

The dependencies of an object under test are automatically initialized by the corresponding InjectionProvider.
These dependencies can also injected into the testcase by using the corresponding injection annotation.

\section{EasyMock}

The EasyMockProvider creates "Nice" Mock objects by default. Such nice mocks allow all method calls and returns appropriate empty values e.g. 0, null or false.
If needed, all mocks can also be converted to use another policy by calling resetAllToNice(), resetAllToDefault() or resetAllToStrict().

The EasyMockProvider implementation is a subclass of EasyMockSupport. EasyMockSupport is a class that meant to be used as a helper or base class to your test cases.
It will automatically register all created mocks and to replay, reset or verify them in batch instead of explicitly.

The following test illustrates the usage of EasyMock with Needle and the injection of generated mock objects.

\begin{lstlisting}[language={JAVA},caption=Testing with EasyMock]
public class AuthenticatorTest {

   @Rule
   public NeedleRule needleRule = new NeedleRule();

   @ObjectUnderTest
   private Authenticator authenticator;

   @Inject
   private EasyMockProvider mockProvider;

   @Inject
   private UserDao userDaoMock;

   @Test
   public void testAuthenticate() throws Exception {
      final User user = new UserTestdataBuilder().build();
      final String username = "username";
      final String password = "password";

      EasyMock.expect(userDaoMock.findBy(username, password)).andReturn(user);

      mockProvider.replayAll();
      boolean authenticated = authenticator.authenticate(username, password);
      Assert.assertTrue(authenticated);
      mockProvider.verifyAll();
   }
}
\end{lstlisting}

EasyMock is the default mock provider. Only the EasyMock library must be added to the test classpath.

For more details about EasyMock, please refer to the EasyMock [http://easymock.org] documentation.

\section{Mockito}

Needle has also an mock provider implementation for Mockito.
Mockito generates Mock objects, where by default the return value of a method is null, an empty collection or the appropriate primitive value.

The following test illustrates the usage of Mockito with Needle.

\begin{lstlisting}[language={JAVA},caption=Testing with Mockito]
public class AuthenticatorTest {
   @Rule
   public NeedleRule needleRule = new NeedleRule();

   @ObjectUnderTest
   private Authenticator authenticator;

   @Inject
   private UserDao userDaoMock;

   @Test
   public void testAuthenticate() throws Exception {

      final User user = new UserTestdataBuilder().build();
      final String username = "username";
      final String password = "password";

      Mockito.when(userDaoMock.findBy(username, password)).thenReturn(user);

      boolean authenticated = authenticator.authenticate(username, password);
      Assert.assertTrue(authenticated);

   }
}

\end{lstlisting}

To use Mockito, the mockito provider must be configured in the \verb|needle.properties| file and the mockito library must be present on test classpath.

\begin{lstlisting}[language={JAVA},caption=Mockito configuration]
mock.provider=de.akquinet.jbosscc.needle.mock.MockitoProvider
\end{lstlisting}

For more details about Mockito, please refer to the Mockito [http://mockito.org] documentation.
