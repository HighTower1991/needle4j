%!TEX root = NeedleReference.tex
\chapter{Getting Started}
In this chapter, a very simple user management application is to be tested using Needle.

\section{Sample Application}
\label{sec: Sample Application}

The example consists of two JPA entity classes User and Profile, with a \verb|@OneToOne| relationship between them and two CDI components.

\subsection{User}
\begin{lstlisting}[language={JAVA},caption=The user entity]
@Entity
public class User {
   @Id
   @GeneratedValue(strategy = GenerationType.AUTO)
   private Long id;

   @Column(unique = true, nullable = false)
   private String username;

   @Column(nullable = false)
   private String password;

   @OneToOne(optional = false, cascade = CascadeType.ALL)
   private Profile profile;

   // Getters and setters
   ...
}
\end{lstlisting}

\subsection{Profile}

\begin{lstlisting}[language={JAVA},caption=The profile entity]
@Entity
public class Profile {
   @Id
   @GeneratedValue(strategy = GenerationType.AUTO)
   private Long id;

   @Column(nullable = false)
   private String language;

   // Getters and setters
   ...
}
\end{lstlisting}

\subsection{Data Access Object}

Now we add a simple DAO component to access the user data.

\begin{lstlisting}[language={JAVA},caption=The User DAO component]
public class UserDao {
   @PersistenceContext
   private EntityManager entityManager;

   public User findBy(final String username, final String password) {
      CriteriaBuilder builder = entityManager.getCriteriaBuilder();
      CriteriaQuery<User> query = builder.createQuery(User.class);

      Root<User> user = query.from(User.class);

      query.where(
            builder.and(builder.equal(user.get(User_.username), username)),
            builder.equal(user.get(User_.password), password));

      return entityManager.createQuery(query).getSingleResult();
   }

   public List<User> findAll() {
      CriteriaBuilder builder = entityManager.getCriteriaBuilder();
      CriteriaQuery<User> query = builder.createQuery(User.class);

      return entityManager.createQuery(query).getResultList();
   }
}
\end{lstlisting}

\subsection{Authenticator}
To authenticate a user, the application uses an authenticator component which itself depends on the User DAO.

\begin{lstlisting}[language={JAVA},caption=The authenticator component]
public class Authenticator {
   @Inject
   private UserDao userDao;

   public boolean authenticate(final String username, final String password) {
      User user = userDao.findBy(username, password);

      return user != null ? true : false;

   }
}
\end{lstlisting}

\section{Using Needle with JUnit}
\label{sec: JUnit}

Needle provides JUnit ``Rules'' to extend JUnit. Rules are basically wrappers around test methods.
They may execute code before, after or instead of a test method.

The following example demonstrates hwo to write a simple JUnit Needle test with two rules.
The database rule provides access to the database via JPA and may execute optional database operations, e.g. to setup the initial data.
The Needle rule does the real magic: it scans the test for all fields annotated with \verb|@ObjectUnderTest| and initializes these tested components
by injection of their dependencies. I.e., the UserDao will get the \verb|EntityManager| field injected automatically. Since we provided a database rule
that entity manager will not be a mock, but a ``real'' entity manager.

Supported injections are constructor injection, field injection and method injection.

\begin{lstlisting}[language={JAVA},caption=JUnit User DAO test]

public class UserDaoTest {
   @Rule
   public DatabaseRule databaseRule = new DatabaseRule();

   @Rule
   public NeedleRule needleRule = new NeedleRule(databaseRule);

   @ObjectUnderTest
   private UserDao userDao;

   @Test
   public void testFindByUsername() throws Exception {
      final User user = new UserTestdataBuilder(
            databaseRule.getEntityManager()).buildAndSave();

      User findBy = userDao.findBy(user.getUsername(), user.getPassword());

      Assert.assertEquals(user.getId(), findBy.getId());
   }

   @Test
   public void testFindAll() throws Exception {
      new UserTestdataBuilder(databaseRule.getEntityManager()).buildAndSave();

      List<User> all = userDao.findAll();

      Assert.assertEquals(1, all.size());
   }
}
\end{lstlisting}

\section{Using Needle with TestNG}
\label{sec: TestNG}

Needle also supports TestNG. There are two abstract test cases that may be extended by concrete test classes.

The class \verb|de.akquinet.jbosscc.needle.testng.AbstractNeedleTestcase| scans all fields
annotated with \verb|@ObjectUnderTest| and initializes the components.

The class \verb|de.akquinet.jbosscc.needle.testng.DatabaseTestcase| can either be used as a special provider for \verb|EntityManager| injection or as a base test case for JPA tests.
In the first case, a new DatabaseTestcase instance is passed to the constructor of the AbstractNeedleTestcase:

\begin{lstlisting}[language={JAVA},caption=TestNG User DAO test]
public class UserDaoTest extends AbstractNeedleTestcase {
  public UserDaoTest() {
    super(new DatabaseTestcase());
  }

  @ObjectUnderTest
  private UserDao userDao;

  @Test
  public void testFindByUsername() throws Exception {
    final User user = new UserTestdataBuilder(getEntityManager())
        .buildAndSave();

    User findBy = userDao.findBy(user.getUsername(), user.getPassword());
    Assert.assertEquals(user.getId(), findBy.getId());
  }

  @Test
  public void testFindAll() throws Exception {
    new UserTestdataBuilder(getEntityManager()).buildAndSave();

    List<User> all = userDao.findAll();
    Assert.assertEquals(1, all.size());
   }
}
\end{lstlisting}
